\section{Notación}
\begin{itemize}
    \item $i.i.d.$: 'Independientes e idénticamente distribuídas'.
    \item $X \indep Y$: "La variable $X$ es independiente a la variable $Y$".
    \item $(E,d)$: Espacio métrico $E$, con métrica $d$.
    \item $\mathcal{B}(E)$: Conjunto de borelianos de $E$.
    \item $C_b (E) := \{f:E \rightarrow R \,| \text{ f es contínua y acotada}\}$
    \item $\mathcal{P}(E)$: Espacio de las medidas de probabilidades sobre $E$.
    \item $c.s.$: "Casi seguramente"
    \item \textbf{T.C.D.}: "Teorema de convergencia  dominada".
    \item $\mathcal{N}(\mu,\sigma)$: "Distribución normal de media $\mu$ y desviación estándar de $\sigma$", se hará un abuso de notación ya que también simbolizará la variable aleatoria que distribuya de esta forma.
    \item $M.M.C.$: "Método de Monte Carlo".
    \item $C.M.H.$: "Cadena de Markov homogénea".
    \item $M.C.M.C.$: "Markov chain Monte Carlo".
    \item $M.B.$: "Movimiento Browniano".
    \item \textbf{t.d.p.}: "Tiempo de parada".
\end{itemize}